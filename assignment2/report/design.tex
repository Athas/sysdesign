\documentclass[12pt]{article}

\author{
        Troels, Troels, Kasper
}
\date{\today}

\title{Design}

\begin{document}

\maketitle

\section{Introduction}
This is time for all good men to come to the aid of their party!


The remainder of this article is organized as follows.
Section~\ref{previous work} gives account of previous work.  Our new
and exciting results are described in Section~\ref{results}.  Finally,
Section~\ref{conclusions} gives the conclusions.

\section{Choice of Language}

A fundamental, and early, design choice is selection of implementation
language.  Our choice is constrained by several factors, in roughly
descending order of priority:

\begin{description}
\item[Existing familiarity: ] We have a hard deadline on
  implementing the design, and the allotted period of time is not
  particularly long.  Notably, we do not have a significant amount of
  time to invest in learning new languages, so we must all be somewhat
  familiar with the language, or at least able to swiftly pick up the
  necessary knowledge.
\item[Availability of FUSE bindings: ] No matter the language, we will
  most likely not have time to develop (and debug) a new set of
  bindings to FUSE.  The language we choose must thus have an existing
  (mature) binding.
\item[Development efficiency: ] As mentioned above, we do not have a
  lot of time to write the code, so the language choice should be
  optimised for developer productivity over such things as final
  program performance.
\item[Implementation portability: ] According to the assignment,
  portability to OS X and GNU/Linux is a goal.  Hence, our chosen
  language must have somewhat mature implementations on both of these
  systems.
\item[Safety: ] A filesystem is a central and important construction,
  and it is important that it is reliable and correct.  We will likely
  not have time to do heavy testing of our work, and it is therefore
  desirable that our chosen language provides as many static
  guarantees of correctness as possible.
\end{description}

Based on these parameters, we have chosen Python.  We are all somewhat
familiar with the language, and we have faith in the maturity of its
FUSE bindings due to the explicit recommendation of Python by the
course lecturers.  Python is also generally acknowledged as a
productive language, and the reference implementation (CPython) is
ported to all relevant platforms.  Python does suffer a bit in the
area of static safety guarantees, though arguably less so than C
(another obvious possibility), but significantly more than languages
such as Haskell or OCaml.

We assume that the possibility of using the Bourne shell scripting
language is a somewhat morbid joke.

\section{Previous work}\label{previous work}
A much longer \LaTeXe{} example was written by Gil~\cite{Gil:02}.

\section{Results}\label{results}
In this section we describe the results.

\section{Conclusions}\label{conclusions}
We worked hard, and achieved very little.

\bibliographystyle{abbrv}
\bibliography{main}

\end{document}
