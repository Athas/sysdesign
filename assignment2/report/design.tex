\documentclass[12pt]{article}
 
\author{
        Troels, Troels, Kasper
}
\date{\today}
 
\title{Design}
 
\begin{document}
 
\maketitle
 
\section{Introduction}
This is time for all good men to come to the aid of their party!
 
 
The remainder of this article is organized as follows.
Section~\ref{previous work} gives account of previous work. Our new
and exciting results are described in Section~\ref{results}. Finally,
Section~\ref{conclusions} gives the conclusions.

\section{Problem analysis}

Our design constraints are heavily influenced by the very specific use
case outlined in the problem assignment.  For clarity, we shall
outline the assumptions on which we have created our design.

\begin{description}
\item[Malcolm is a novice user,] and he will likely not understand
  complex versioning operations like branching or tagging.
  Additionally, he is the only one using the system: there is no need
  for a notion of multi-user support, or synchronisation against a
  remote/central repository.
\item[The versioning system is for documents,] not arbitrary files
  managed by the operating system (such as logs, for example).  This
  also implies that performance is not a chief concern.
\item[The central purpose is preventing data loss,] so it is more
  important to ensure that older versions can be restored than to
  provide advanced facilities for inspecting the history of a file.
  The ability to inspect a log of changes to a file is only important
  insofar as it helps the user find the last revision containing his
  missing data.
\item[Portability is somewhat of a concern,] as the assignment
  mandates we use FUSE, ostensibly for reasons of portability.
\item[The system has to be transparent,] and we cannot ask Malcolm to
  alter his usual work routine in any way.  When he clicks the
  \texttt{Save}-button in his program, we have to make sure older
  revisions of the working document are not lost.
\end{description}
 
\section{Choice of Language}
 
A fundamental, and early, design choice is selection of implementationlanguage. Our choice is constrained by several factors, in roughly
descending order of priority:
 
\begin{description}
\item[Existing familiarity: ] We have a hard deadline on
  implementing the design, and the allotted period of time is not
  particularly long. Notably, we do not have a significant amount of
time to invest in learning new languages, so we must all be somewhat  familiar with the language, or at least able to swiftly pick up the
  necessary knowledge.
\item[Availability of FUSE bindings: ] No matter the language, we will  most likely not have time to develop (and debug) a new set of
  bindings to FUSE. The language we choose must thus have an existing
  (mature) binding.
\item[Development efficiency: ] As mentioned above, we do not have a
  lot of time to write the code, so the language choice should be
  optimised for developer productivity over such things as final
  program performance.
\item[Implementation portability: ] According to the assignment,
  portability to OS X and GNU/Linux is a goal. Hence, our chosen
  language must have somewhat mature implementations on both of these
  systems.
\item[Safety: ] A filesystem is a central and important construction,
  and it is important that it is reliable and correct. We will likely
  not have time to do heavy testing of our work, and it is therefore
  desirable that our chosen language provides as many static
  guarantees of correctness as possible.
\end{description}
 
Based on these parameters, we have chosen Python. We are all somewhat
familiar with the language, and we have faith in the maturity of its
FUSE bindings due to the explicit recommendation of Python by the
course lecturers. Python is also generally acknowledged as a
productive language, and the reference implementation (CPython) is
ported to all relevant platforms. Python does suffer a bit in the
area of static safety guarantees, though arguably less so than C
(another obvious possibility), but significantly more than languages
such as Haskell or OCaml.
 
We assume that the possibility of using the Bourne shell scripting
language is a somewhat morbid joke.
 
\section{Choice of version control system}
The file system we are supposed to end up with should only support a
single user.
The primary goal is to allow the user to go back to an earlier version
of any file. It's not suggested that the user should have several
different versions of the same file.

By using these specifications we have to decide whether to implement
our own version control or use an existing such as Subversion or Git.

Using an already existing version control system we not have to
implement any actual data-compression or logic for handling revisions.
Instead we would have to create an interface for the chosen version
control system. This seems to be the easier choice, but we think the
version control systems that exist are too complex for the simple
needs of this file system. Implementing our own system has the
advantage of not adding more complexity than needed.

Branches aren't needed as the user is not supposed to have multiple
versions of the same file active. Therefore merges aren't needed.
There is no need for collaboration either. All in all git and
subversion, which are the two version control systems we considered,
are too complex, so we decided to implement our own simple system.
 
\section{Feeing up disk space}

To be able to go back to old versions of files we would have to keep
all the old data stored somewhere. After some time most of the disk
space would be filled with old versions of files. This can be
postponed by only recording changes to files, but it would still
happen. Some mechanism to clean out old, unused versions of files must
therefore be implemented.

We have decided that implementing a cleaner-process as discussed in
[Santry99] would be the right solution. As the assignment states that
the file system should only be used in the Documents directory, we
assume that all the files are actually user-created files. We also
assume that all files in the directory are equally important to keep
old revisions of. This allows us to use a simpler policy than
described in the article. Instead of having different policies for
different groups of files we are able to have a global policy for the
entire file system.

\subsection{File retention policies}
We have decided to use a policy that use a mixture of two different
policies, based on time of the revision. If a revision is less than
one month old the Keep All-policy is used. This saves all changes to
the file, which takes up the most space. When a revision is older than
one month the policy changes to Keep Landmarks, which means that
revisions close to each other, time wise are deleted except the
newest. This is useful as the older revisions get the less likely the
user is to know the difference between two adjacent revisions, and the
eldest revision is the most likely to be unneeded.
The one-month limit and the decision about when two files are close to
each other should be changeable by the user, as the need for cleanup
is based on several things such as disk size, importance of files etc.

\subsection{File-locking}
This process has to be able run concurrently with the file system
being in use. The worst case scenario being that a user is trying to
retrieve a version of a file while the cleaner-process deletes it. To
avoid this, the file system and the cleaner-process has to have a way
to tell each other when a file is in use. We have decided to implement
this using the rename system call, as it is promised to be atomic. In
the version-folder for a given file an .unlocked-file is located.
Whenever either the cleaner-process or the file system wants to use
the file, they have to rename .unlock to .lock. This guarantees that
if the rename-call is successful, it is the only process accessing the
file at a given time. If the other process tries to access the same
file it will have to wait until the file is renamed back to unlock
again.

\subsection{Locating files to check}
As the number of files grows a way to identify files likely to be
ready for cleanup is needed. When there is only a small amount of
files the cleaner-process can just check all of them at every
iteration, but as soon as the number of files grow it becomes
unfeasible. A way to identify files that are likely to have revisions
that can be deleted is therefore needed.

To avoid having to scan the file system for files a list of all the
files on the system should be kept in memory. Together with each entry
in the list a date is kept that signify when the cleaner-process
should analyze the file again. This is a simpler implementation of the
way it is done in the Elephant file system \cite{Santry:1999gf}. When
the cleaner-process has analyzed the file a new date is created
according to the findings. The exact intervals at which files should
be analyzed has to be worked out during testing.

\section{File system interface}

A somewhat small decision when designing with enormous impact for the
user is the way that files are shown to the user.
We have to decide on how the user sees the different versions of the
file. A possibility is to show all available versions of all the
files,
which has the advantage that the user can easily see every single
change to all files, but this quickly becomes a problem.
As soon as there are more than a few files or versions it becomes
impossible to get an overview of the files in a directory.
Instead we have decided that it should be completely invisible to the
user that the file system is actually implementing version control.
This makes the file system usable by everybody and only if a user has
to get an earlier revision will he have to know the specifics about
the file system.
We have decided to use semicolon followed by a number or * to use
versions.
When a ls command is issued with ;* it shows all the available
versions of the files in the directory.
Each version has a number attached, so when file is referred to with ;
and a number it references that version of the file.
Because of this we have had to enforce that files can not have a name
with the suffix ;* or semicolon followed by a number.

\section{Internal file representation}

\section{Delta compression}

\section{Link-handling?}

\section{GUI interface?}


\bibliographystyle{abbrv}
\bibliography{assignment1}
 
\end{document}
 