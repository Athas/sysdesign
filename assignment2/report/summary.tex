\documentclass[12pt]{article}
 
\author{
        Troels Henriksen, Troels Visti Thrane, Kasper Middelboe Petersen
}
\date{\today}
 
\title{System Design, assigment 2a:\\Summary of filesystem design}
 
\begin{document}
 
\maketitle
 
\section{Design background}

We are designing a versioning filesystem for managing the documents of
a novice user.  Managing the revision history completely should be
transparent to the user and his programs; the user interface for
restoring past versions is beyond the scope of the design, but we must
provide at least a programming interface to the version history.

The most important issue is preventing data loss, not storing detailed
logs about changes.  In addition, the system is inherently single-user
and runs entirely on the users own machine.

\section{Design requirements}

\begin{description}
\item[FUSE must be used:] this is an externally mandated requirement,
  but still a good choice for our domain.
\item[Versioning must be transparent,] no special commands must be
  necessary to create revisions of files, and no deleted files may be
  visible in directory listings (unless special recovery-precautions
  are taken).
\item[The revision history is read-only,] so we provide no facilities
  for explicitly modifying the revision history.
\end{description}

\section{Design specifics}

Our file system is implemented as a program that translates FUSE calls
into operations on an underlying versioned file storage, said storage
being implemented as a directory structure on a normal file system, as
in \cite{1096690} and \cite{Bustamante04wayback:a}.

Access to older versions is done making modifications to the normal file 
namespace by adding an optional \texttt{;<version number>} suffix to files
and listing directory contents by adding \texttt{;*} suffix to directories.

To conserve disk space, it may be necessary to merge adjacent
revisions, or even delete old revisions outright.  This will be done
by a cleaning process, as in \cite{319159}, although we do not support
configuration of the retention policy on a per-file basis, only on a
global whole-system basis.

Our locking-strategy (to prevent race-conditions) is based on file
renaming, which is an atomic operation.  The same strategy is followed
in.\cite{Santry:1999gf}

\subsection{Implementation language}

Based in particular on the recommendation from the course lecturers,
we have decided to implement the file system driver in Python.
Arguments in favour of this choice include the development efficiency
of Python (important as we only have a single week to implement the
file system), the existence of mature FUSE bindings, as well as some
degree of Python experience among all members of the group.

\section{Development strategy}

Our development strategy is based on dividing the system into the
following functional units:

\begin{itemize}
\item Revision cleaner
\item Saving
\item Reading the newest revision
\item Reading some older revision
\item Listing the contents of a directory
\item Renaming a file
\item Deleting a file
\end{itemize}

\bibliographystyle{abbrv}
\bibliography{assignment1}
 
\end{document}
